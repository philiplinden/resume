%% RESUME - Philip Linden
\documentclass[10pt,final,sans]{resume}

\begin{document}
\setlength\headheight{28pt} % make header tall enough
\name{PHILIP J. LINDEN}
\lcontact{
    \begin{tabular}{@{}cl@{}}
        \faEthereum & \href{https://etherscan.io/address/0x6bFd9e435cF6194c967094959626ddFF4473a836}{philiplinden.eth} \\
        \faGithub & \href{https://github.com/philiplinden/}{philiplinden} \\
        \faLinkedin & \href{https://www.linkedin.com/in/philiplinden/}{philiplinden}
  \end{tabular}
} \rcontact{
    \begin{tabular}{@{}r@{}}
        lindenphilipj@gmail.com \\
        1 Quiet St \\
        North America, Earth 12345 \\
        (585) 690-7067
    \end{tabular}
}

\section{Professional Summary}
I am an engineer who is passionate about the design and analysis of aviation and
space systems, including but not limited to satellites, human spaceflight,
spacecraft and aircraft structures, propulsion, mechanisms, remote sensing,
imaging, and controls. I am relentlessly curious, a strong visionary, and
optimistic about the future of technology and humankind.

\section{Education}
\headerwithlabel{Rochester Institute of Technology}{Rochester, NY}{May 2017}
Bachelor of Science in Mechanical Engineering -- Aerospace Option \\
Master of Engineering in Mechanical Engineering

\section{Technical Skills}
Mission Operations, Space Systems Engineering, Imaging Science, Git, Python, Rust, MATLAB, Simulink, Docker, CAD,
{\textrm \LaTeX}, MacOS, Linux, Controls, Web3 (Ethereum), Timekeeping, Technical Writing, Public Speaking

\section{Experience}

\headerwithlabel{Planet}{San Francisco, CA}{November 2018 -- Present}
Senior Space Systems Engineer, Mission Operations
\begin{itemize}
    \item Designed and implemented automated flight operations scripts to maximize satellite operations uptime, detect and respond to anomalies, and update onboard software.
    \item Responsible Engineer for the entire SkySat optical assembly and onboard imaging chain.
    \item Responsible Engineer for SkySat, Pelican, and Tanager imaging ConOps and on-orbit payload commissioning.
\end{itemize}

Space Systems Engineer / Flight Operator, SkySat Mission Operations
\begin{itemize}
    \item Conducted flight operations for the SkySat constellation of 21 Earth
    observation satellites, including manual commanding, anomaly investigation \&
    resolution, and development of new spacecraft activities, operations
    procedures, automated procedures, and automated on-orbit activities.
    \item Maintained spacecraft testbed and ground support equipment for
    hardware-in-the-loop testing.
    \item {\bf Publications:} \href{https://digitalcommons.usu.edu/smallsat/2021/all2021/189/}{SSC21-VIII-05} (2021), \href{https://doi.org/10.1109/AERO55745.2023.10115608}{doi:10.1109/AERO55745.2023.10115608} (2023)
\end{itemize}

\headerwithlabel{MoonDAO}{San Francisco, CA}{January 2023 -- Present}
Rocketeer, Senator, Citizen Voter
\begin{itemize}
    \item Voting member, community manager, and among the top 5 most active contributors.
    \item {\bf Project:} DeSci Labs Publication Reproducibility Validations
    \item {\bf Project:} Cislunar Open Clock Synchronization System (CLOCSS)
\end{itemize}

\headerwithlabel{Open Lunar Foundation}{San Francisco, CA}{January -- May 2023}
Research Fellow, Timekeeping \& Lunar Clocks
\begin{itemize}
    \item Researched the feasibility of a local lunar time standard or a shared and openly accessible reference timing signal for positioning, navigation, and timing (PNT) capabilities of lunar missions.
    \item Proposed the concept of a local lunar time standard that can be accessed using technology that is likely to be included in most lunar missions for nominal activities.
    \item {\bf Project:} Possibilities for a Local Lunar Time Standard
\end{itemize}

\headerwithlabel{Lockheed Martin Space}{Sunnyvale, CA}{June 2017 -- November 2018}
Electro-Optical Engineer, Optical Payload Center of Excellence
\begin{itemize}
  \item Characterized focal plane arrays and imaging systems in optical labs.
  \item Systems engineering and Electro-Optical engineering on IRAD projects to
  support major business pursuits.
  \item Led a software team through critical development milestones for Matlab
  engineering tools.
\end{itemize}
\headerwithlabel{SpaceX}{Hawthorne, CA}{June -- August 2016}
Vehicle Engineering Intern, Capsule Structures
\begin{itemize}
  \item Modeled and drafted designs for critical structures for the Crew Dragon
  vehicle.
\end{itemize}

Vehicle Engineering Intern, Capsule Reusability \hfill {\it January -- July 2015}
\begin{itemize}
  \item Project development, including hands-on prototyping and designing,
  conducting and \\
  presenting experiments to explore changes to Dragon Cargo space capsules.
\end{itemize}

\headerwithlabel{RIT Space Exploration (RITSPEX)}{Rochester, NY}{Fall 2014 -- Present}
Alumni Member
\begin{itemize}
  \item Mentored undergraduate students working on space exploration projects.
  \item Provided subject-matter expertise in imaging projects and control systems.
\end{itemize}

\break
\section{Detailed Project Descriptions}
\headerwithlabel{spacetime}{(work in progress)
\href{https://philiplinden.github.io/spacetime/}{philiplinden.github.io/spacetime}}{Personal,
2024} A simulation of heterogeneous networked clocks in cislunar space, built in
Rust on the Bevy game engine. This project aims to use agent-based modeling and
simulations to explore how network topology and time synchronization evolve in a
growing lunar ecosystem. Modeling how the approach may be used in practice and
at scale informs decisions about how to architect a robust cislunar PNT system
that scales with the population.


\headerwithlabel{DeSci Reproducibility
Validations}{\href{https://github.com/philiplinden/cremons-et-al-2022}{\faGithub\
philiplinden/cremons-et-al-2022} \quad
\href{https://nodes.desci.com/dpid/137}{nodes.desci.com/dpid/137}}{MoonDAO, 2023}
Reproduced figures and findings of \href{https://doi.org/10.1029/2022/EA002277}{doi:10.1029/2022/EA002277} in
Matlab and converted code to Python. Published on DeSci Labs with data, code,
commentary, and the original manuscript.

\headerwithlabel{Cislunar Open Clock Synchronization System
(CLOCSS)}{\href{https://www.youtube.com/watch?v=cd8hiubLy48}{presentation}}{MoonDAO,
2023} {\it DARPA LunA-10 Proposal (Shortlisted)} Developed a concept for a
decentralized approach to lunar infrastructure. Authored manuscript and
presentation materials submitted to DARPA's
\href{https://www.darpa.mil/news-events/2023-08-15}{LunA-10} capability study.

\headerwithlabel{Possibilities for a Local Lunar Time
Standard}{\href{https://www.openlunar.org/research/possibilities-for-a-local-lunar-time-standard}{white
paper}}{Open Lunar Fellowship, 2023} Authored a white paper that explores the
characteristics of a common reference timing signal to serve future lunar
operations. The goal was to identify a low-cost, transparent approach to the
development of a Local Lunar Time Standard.

\headerwithlabel{On-Orbit Demonstrations of Proactive Tasking of Glint Imagery}{}
{IEEE Aerospace Conference, 2023}
{\it Awarded Best Paper in Track (Track 12)} \href{https://doi.org/10.1109/AERO55745.2023.10115608}{doi:10.1109/AERO55745.2023.10115608} \\
Formalized a methodology to predict future glint windows over a specific region.
Studied various tasking approaches that described the satellite's actions during
these windows to autonomously acquire glint captures. These actions were then
demonstrated by orbiting satellites, and their captures were then analyzed.

\headerwithlabel{Automatic Optical Image Stabilization System Calibration}
{\href{https://digitalcommons.usu.edu/smallsat/2021/all2021/189/}{SSC21-VIII-05}}
{Small Satellite Conference, 2021} Developed and executed an on-orbit
calibration campaign. Automated on-orbit procedures and analyses were used for
calibration and validation of an optical image stabilization (OIS) system across
a fleet of 19 satellites. OIS actuation settings were configured for each image
capture through automated optimization procedures.

\headerwithlabel{High Altitude Balloon Autonomous Altitude Control
System}{\href{https://brickworks.github.io/Nucleus/pdr_altitudecontrol/}{brickworks.github.io/Nucleus}}{Personal, 2020}
Designed a control system for a high altitude balloon (HAB)
to maintain a target altitude by venting gas from the balloon and dropping
ballast mass in flight. I modeled passive flight dynamics of a HAB in MATLAB,
Simulink, and Python. I developed a state machine and control system in
Simulink and used the model to tune a PID controller. I then derived a
state-space model, Kalman filter, and LQR controller from scratch to achieve
better performance.

\headerwithlabel{Cosmic Dawn Intensity Mapper (CDIM)}{\href{https://ui.adsabs.harvard.edu/link_gateway/2019BAAS...51g..23C/doi:10.48550/arXiv.1903.03144}{doi:10.48550/arXiv.1903.03144}}{Graduate Paper, 2017}
Contributed to a proposal for a Probe Class (\textasciitilde\$850M) NASA mission
for a 1.5 meter space telescope intended to observe near-infrared light from the
early universe. Compiled financial, mass, and power budgets for the optics,
instruments, cryocooler \& spacecraft. Defined system-level design, generated
representative CAD models and figures of the spacecraft. This mission was
published in the NASA 2020 Decadal Survey. My contribution: \href{https://github.com/runphilrun/CDIM-design/blob/master/cdim_design.pdf}{\it \faGithub\ runphilrun/CDIM-design}

\headerwithlabel{1 kW Arcjet Thruster}{\href{https://github.com/RIT-Space-Exploration/msd-P17101/blob/master/p17101.pdf}{\faGithub\ RIT-Space-Exploration/msd-P17101}}{Undergraduate Capstone, 2017}
Developed the concept, system-level design, and nozzle design for a small scale
arcjet thruster demonstration. Worked in a multidisciplinary team of mechanical
and electrical engineers. Responsible for communication between the team and the
customer (RIT Space Exploration). Designed and performed CFD analysis on the
thruster nozzle.

\headerwithlabel{SPEXcast Podcast}{\href{https://blog.spexcast.com/}{blog.spexcast.com}}{Personal, 2016--2021}
I produce, edit, and co-host a space exploration podcast, which is a weekly
discussion podcast the science and technology of space exploation. SPEXcast
also features interviews with space scientists and industry members, including
Tory Bruno, Chris Hadfield, and NASA Scientists.

\section{Other Skills}
Brazilian Portuguese - Intermediate Level (written \& spoken)

\end{document}
